\usepackage[utf8]{inputenc}          % кодировка
\usepackage[T1,T2A]{fontenc}         % тоже кодировка
\usepackage[english,russian]{babel}  % языки

\usepackage{multirow}
\usepackage{tabularx,booktabs}

% \usepackage{ulem}   % этот пакет используется ниже с другими параметрами (хз, зачем он был здесь)

\usepackage{graphicx, float}   %пакет для единого оформления всех плавающих объектов (избегаем повторяющихся команд в документе)

\DeclareGraphicsExtensions{.pdf,.png,.jpg,.eps}   % форматы
\usepackage{titlesec}
\setcounter{secnumdepth}{4}
\titleformat{\paragraph}
{\normalfont\normalsize\bfseries}{\theparagraph}{1em}{}
\titlespacing*{\paragraph}
{0pt}{3.25ex plus 1ex minus .2ex}{1.5ex plus .2ex}

\graphicspath{{images/}}   % выбираем папку, в которую сохраняем все рисунки, чтобы не было хаоса файлов

\usepackage{amsmath,amssymb}   % математические формулы и символы

\usepackage[a4paper,left=25mm,right=15mm,top=20mm,bottom=20mm]{geometry}   % устанавливает поля документа

\parindent=0ex            % красная строка
\parskip=5mm              % расстояние между параграфами
\usepackage{indentfirst}  % делать отступ в начале параграфа

\usepackage{hyperref}   % добавление ссылок

\def\hmath$#1${\texorpdfstring{{\rmfamily\textit{#1}}}{#1}}   % настройка подписей плавающих объектов

\makeindex   % нумерация

\usepackage{array,graphicx,caption}   % картинки, подписи, таблицы
%\usepackage{endfloat} - для вывода картинок со списком в конце файла

\usepackage{caption}   % подписи к картинкам

\usepackage[labelformat=simple]{subcaption}     % для subfigure
\renewcommand\thesubfigure{(\alph{subfigure})}

\usepackage[export]{adjustbox}   % чтобы влево-вправо картинки ставить

\usepackage[labelformat=simple]{subcaption}
% метка subfigure: "(а)" вместо дефолтного "а"

\renewcommand\thesubfigure{(\alph{subfigure})} % для продвинутого captionof

\usepackage{afterpage,placeins} % для барьеров

\usepackage{wrapfig} %добавление wrapfig

\usepackage[nottoc]{tocbibind} %подключает в содержание список лит-ры

\usepackage{multicol}

\usepackage{listings}

\usepackage[normalem]{ulem}
\usepackage{verbatim}

\usepackage{xcolor}

%New colors defined below
\definecolor{codegreen}{rgb}{0,0.6,0}
\definecolor{codegray}{rgb}{0.5,0.5,0.5}
\definecolor{codepurple}{rgb}{0.58,0,0.82}
\definecolor{backcolour}{rgb}{0.95,0.95,0.92}

%Code listing style named "mystyle"
\lstdefinestyle{mystyle}{
  backgroundcolor=\color{backcolour}, commentstyle=\color{codegreen},
  keywordstyle=\color{magenta},
  numberstyle=\tiny\color{codegray},
  stringstyle=\color{codepurple},
  basicstyle=\ttfamily\footnotesize,
  breakatwhitespace=false,
  breaklines=true,
  captionpos=b,
  keepspaces=true,
  numbers=left,
  numbersep=5pt,
  showspaces=false,
  showstringspaces=false,
  showtabs=false,
  tabsize=2
}

%"mystyle" code listing set
\lstset{style=mystyle}